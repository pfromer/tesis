\documentclass[11pt,a4paper,twoside]{tesis}
% SI NO PENSAS IMPRIMIRLO EN FORMATO LIBRO PODES USAR
%\documentclass[11pt,a4paper]{tesis}

\usepackage{graphicx}
\usepackage[utf8]{inputenc}
\usepackage[spanish]{babel}
\usepackage[left=3cm,right=3cm,bottom=3.5cm,top=3.5cm]{geometry}

\begin{document}

%%%% CARATULA

\def\autor{Pablo Víctor Fromer}
\def\tituloTesis{Datalog +/-, \vspace{.2cm} \\ una interfaz tolerante a la inconsistencia }
\def\runtitulo{Datalog +/-, una interfaz tolerante a la inconsistencia}
\def\runtitle{Datalog +/-, una interfaz tolerante a la inconsistencia}
\def\director{Vanina Martinez}
\def\codirector{Ricardo Rodriguez}
\def\lugar{Buenos Aires, 2019}
\input{caratula}

%%%% ABSTRACTS, AGRADECIMIENTOS Y DEDICATORIA
\frontmatter
\pagestyle{empty}
\input{abs_esp.tex}

\cleardoublepage
\input{abs_en.tex} % OPCIONAL: comentar si no se quiere

\cleardoublepage
\input{agradecimientos.tex} % OPCIONAL: comentar si no se quiere

\cleardoublepage
\input{dedicatoria.tex}  % OPCIONAL: comentar si no se quiere

\cleardoublepage
\tableofcontents

\mainmatter
\pagestyle{headings}

%%%% ACA VA EL CONTENIDO DE LA TESIS

\chapter{Introducción}
\section{Motivación Opcion 1}

Los lenguajes ontológicos, los sistemas basados en reglas y sus integraciones han jugado un rol central en el desarrollo de la Web Semántica…  Antes del surgimiento de Datalog +/- no existía literatura que explique cómo generalizar reglas y dependencias de bases de datos de manera que puedan expresar axiomas ontológicos. De esta manera no estaba cubierto el interés en la comunidad de la Web Semántica por conseguir formalismos altamente escalables que pudieran hacer que la red se beneficie de las tecnologías ya implementadas en los motores de bases de datos.

Datalog +/- es una serie de variantes del lenguaje Datalog que son particularmente adecuadas para responder consultas sobre ontologías de manera eficiente. Estas variantes extienden Datalog con la posibilidad de cuantificar existencialmente en las implicaciones de las reglas y otra serie de ventajas, mientras que al mismo tiempo restringen el lenguaje de manera sintáctica con el objetivo de asegurar tratabilidad.

Por otro lado, tanto en el campo de las bases de datos como en el de la Web Semántica es ampliamente reconocido que la inconsistencia es un problema que no puede ser ignorado. Al momento de integrar información proveniente de diversas fuentes, ya sea para popular una ontología o para responder una consulta, las restricciones de integridad son muy propensas a ser violadas en la práctica. En este trabajo proponemos una implementación de la semántica IAR, la cual describe un método para tratar ....

Finalmente presentamos una herramienta que tiene el propósito de acercar a la comunidad una implementación de Datalog+/-, con el objetivo de poder acceder a este lenguaje de manera fácil, a través de Internet...

\section{Datalog +/-}
\section{Consultas y bases de datos}

Con respecto a los ingredientes elementales, se asumen constantes, nulos y variables de la siuguiente manera; estos son los arguentos de las fórmulas atómicas en las bases de datos, consultas y dependencias.
Se asume (i) un universo infinito de constantes  $\Delta$ (las cuáles forman el dominio de la base de  datos), (ii) un conjunto infinito de nulos etiquetados $\Delta_{N}$ (representando valores desconocidos), y (iii) un conjunto infinito de variables X (que se utilizan en las consultas y en las dependencias). Cada constante representa un valor diferente, mientras que nulos distintos podrían representar el mismo valor. 
Se asume un orden lexicográfico en  $\Delta$ $U$  $\Delta_{N}$, donde los símbolos de $\Delta_{N}$ siguen a los de $\Delta$. Se denota por $X$ a una sequencia de variables $X_{1}$, ..., $X_{K}$ con k > 0.



%% ...
\chapter{Resultados Formales}
\chapter{Implementación}

%%%% BIBLIOGRAFIA
\backmatter
%\bibliography{tesis}

\end{document}
